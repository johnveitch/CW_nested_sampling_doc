\section{Evaluating the code}

In this section we will show how the posterior parameter distributions that we can generate from the output
of our nested sampling code compare to evaluating them with a previous implementation of the analysis code
\citep[\lppe used in e.g.][]{2014ApJ...785..119A} using a (generally inefficient) MCMC algorithm and over a grid in parameters.
We will show comparisons of the codes in two different regimes: no signal is present in the data, and an obvious signal is
present in the data. We do this using simulated data both containing purely Gaussian noise and containing simulated signals
added to Gaussian noise.

Using a selection of many simulated signals we also show how the odds comparing a signal model and a noise only model change with
signal-to-noise ratio, along with comparing coherent verses incoherent signals (for coherent simulated signal and incoherent simulations).
The simulations also allow us to evaluate the posterior probabilty distributions produced by the code using what is colloquially known,
in ground-based \gw parameter estimate circles at least, as ``P-P plots'' (see \S\ref{sec:ppplots})

\subsection{Code comparison}

In each of the tests below we will be running our code (\lppen) using the default proposal distributions discussed in \S\ref{sec:proposals}
and with the number of live points fixed to be 2048. We will also assume a source at a fixed sky position and that the signal model is purely
that for the $l=m=2$ harmonic (Equation~\ref{eq:h2f}), and that we work in terms of the signal amplitude $h_0$ and signal phase
$\Phi_{22}^C$ (this is for consistency with the old code which uses $\Phi_{22}^C$ rather than $\phi_0$, which we have earlier defined as the
rotational phase). The simulated data in all cases 
will be generated as the complex heterodyned
time series sampled once per minute. When comparisons between the current code and previous codes are 
made we run the current code in a way that splits datasets into 30 point chunks, rather than using the algorithm described in \S\ref{sec:splitting},
as this makes it consistent with the old code (\lppe). We also purely use the uniform prior ranges for the amplitude parameter to be
consistent between codes.

\subsubsection{Simulated noise}\label{sec:simnoise}

An initial test of the code is whether the output posterior probability distributions match those produced when evaluating the
posterior over a fixed grid in the four-dimensional parameter space of $\vec{\theta} = \{h_0, \cos{\iota}, \psi, \phi_{22}^C\}$, when
using purely Gaussian noise.\footnote{It is worth noting that in many of these tests the polarisation angle prior covers $-\pi/4$ to $\pi/4$
rather than the 0 to $\pi/2$ range given in, e.g., \citet{2015MNRAS.453.4399P} due to this being the range required by the older
parameter estimation MCMC code. Although this range is degenerate to rotations over $\pi/2$, so spans the same range of waveform models.}
A comparison of the marginalised posteriors for individual parameters, and pairs of parameters, are shown when using simulated data lasting
ten days from a single detector (in this case assumed to be the LIGO Hanford detector, H1) in Figure~\ref{fig:simnoise_single}.

It can be seen that the output posteriors look qualititavely consistent. However, we can also quantify some aspects of consistency by
comparing the evidence output from the nested sampling code with that estimated from the grid-based method from \lppe, the upper limits
on $h_0$ produced by the codes, and performing Kolmogorov-Smirnov consistency tests between the nested sampling posterior samples
and MCMC posterior samples (giving a $p$-value for the null hypothesis that the samples are from the same distributions).
These are shown in Table~\ref{tab:codeeval} and show very good consistency between the codes, although it should be noted that
these are for one particular run and some statistical fluctuations in the exact values for different runs are to be expected (see e.g.\
the variations in evidence values in \S\ref{sec:proposaltesting}).

A {\tt jupyter} notebook with this test can be found \href{https://github.com/mattpitkin/CW_nested_sampling_doc/blob/master/figures/codeeval/simulations/noise/SimulatedNoiseTestsPaper.ipynb}{here}.

\begin{table*}[hptb]
\caption{Consistency tests between outputs of the new code, \lppen, and the old code, \lppe, when running on simulated data
and searching over the four parameters $\{h_0, \cos{\iota}, \psi, \Phi_{22}^C\}$.\label{tab:codeeval}}
\begin{center}
\begin{tabular}{l c c | c c c c}
\hline
\multirow{2}{*}{Simulation} & \multirow{2}{*}{$\ln{\left(\frac{Z_{\text{nested}}}{Z_{\text{grid}}}\right)}$} & \multirow{2}{*}{$\frac{(h_0^{95\%})_{\text{nested}}}{(h_0^{95\%})_{\text{grid}}}$} & 
\multicolumn{4}{c}{K-S $p$-value} \\ \cline{4-7}
 &  &  & $p(h_0)$ & $p(\Phi_C^{22})$ & $p(\cos{\iota})$ & $p(\psi)$ \\                      
\hline
\hline
Noise (single detector)  & $-0.07$ & $1.009$ & 0.404 & 0.026 & 0.010 & 0.703 \\
Noise (two detectors)    & $-0.05$ & $0.992$ & 0.141 & 0.564 & 0.538 & 0.493 \\
Signal (single detector) & 0.245   & $1.004$ & 0.237 & 0.291 & 0.125 & 0.175 \\
Signal (two detectors)   & 0.240   & 0.991   & 0.722 & 0.052 & 0.067 & 0.032 \\
\hline
\end{tabular}
\end{center}
\end{table*}

We see a very similar situation, in terms of agreement between the codes, when running on simulated data assumed to be from two detectors (the LIGO
H1 and L1 sites) as shown in Figure~\ref{fig:simnoise_single} and the second line of Table~\ref{tab:codeeval}. A {\tt jupyter} notebook with this test
can be found \href{https://github.com/mattpitkin/CW_nested_sampling_doc/blob/master/figures/codeeval/simulations/noise_multidet/SimulatedNoiseTestsMultidetPaper.ipynb}{here}.

The above tests using simulated noise and with searches covering the four standard unknown \gw parameter show that the code appears to
be working as expected, and is in agreement with our previously used code. However, more generally we can perform parameter estimation
over a larger number of signal parameters, and can again check for consistency with the previous code. We produce one day of Gaussian
noise for a single detector and search over the four \gw parameter with the same priors as before, but then also use a multi-variate Gaussian prior (\S\ref{sec:gaussianprior})
over the {\it phase} parameters: rotational frequency and first frequency derivative, and binary parameters of binary period, time of periastron, angle of
periastron and its first derivative, projected semi-major axis, and eccentricity ($\{f,\dot{f},P_{\text{b}}, T_0, \omega_0, \dot{\omega}_0, a\sin{i}, e\}$).
This gives a total of 12 parameters in the search. The multi-variate prior is defined such that all parameters are uncorrelated except, in this case,
the parameter pairs $[T_0, \omega_0]$ and $[P_{\text{b}}, \dot{\omega}_0]$ for which we use a very high correlation of $0.9999$ (we do not set them to
be fully correlated due to numerical issues inverting such matrices).

The one-and-two dimensional posteriors output for this case when run with both \lppe and \lppen can be seen in Figure~\ref{fig:noise_multiparam},
which also overlays the marginalised priors on top. It can be seen qualititavely that the codes are consistent\footnote{It is worth mentioning that in performing
these tests a bug was discovered in \lppe in which Equation~\ref{eq:deltaphi} was being applied with the wrong sign, leading to parameter estimates
having the wrong sign. However, this bug was fixed for the example shown here.} and for the {\it phase} parameter priors are recovered, except for
$\dot{f}$, which has some structure due to the specifics of the noise realisation. The Kolmogorov-Smirnov test $p$-values testing the null hypothesis
that the posterior samples from each code are drawn from the same distribution are shown in Table~\ref{tab:noisemultiks}, and suggest that the
distributions are consistent.

\begin{table*}[hptb]
\caption{Kolmogorov-Smirnov test $p$-values testing the null hypothesis that the samples output by \lppen and \lppe, when running on simulated
noise data and searching over the twelve parameters $\{h_0, \cos{\iota}, \psi, \Phi_{22}^C, f,\dot{f},P_{\text{b}}, T_0, \omega_0, \dot{\omega}_0, a\sin{i}, e\}$,
are drawn from the same distributions.\label{tab:noisemultiks}}
\begin{center}
\begin{tabular}{l | c c c c c c c c c c c c}
\hline
Parameter & $h_0$ & $\Phi_C^{22}$ & $\cos{\iota}$ & $\psi$ & $f$ & $\dot{f}$ & $P_{\text{b}}$ & $T_0$ & $\omega_0$ & $\dot{\omega}_0$ & $a\sin{i}$ & $e$ \\                      
\hline
\hline
$p$-value  & 0.13 & 0.04 & 0.78 & 0.23 & 0.44 & 0.26 & 0.85 & 0.28 & 0.29 & 0.86 & 0.52 & 0.28 \\
\hline
\end{tabular}
\end{center}
\end{table*}

\subsubsection{Simulated signals}\label{sec:simsignal}

The next test is checking how the codes compare when there is a signal present in the data. When generating the simulated signals we initially
assume that the signal's phase evolution is perfectly known and has been removed via the heterodyne described in \S\ref{sec:model}, and therefore
$\Delta\phi_2 = 0$ from Equation~\ref{eq:deltaphi}. As in \S\ref{sec:simnoise} then will search over the four-dimensional parameter space of
$\vec{\theta} = \{h_0, \cos{\iota}, \psi, \Phi_{22}^C\}$.

We create a simulated signal in the H1 detector spanning ten days of data with parameters ($h_0 = 6.2\ee{-23}$, $\Phi_{22}^C = 2.4$\,rad, $\cos{\iota} = 0.3$
and $\psi = 0.1$\,rad) that produce a signal-to-noise ratio of 8 when injected into
Gaussian noise. The posterior probability distributions estimated for the parameters (expressed as offsets from the assumed heterodyne parameter values) are
shown in Figure~\ref{fig:simsignal_single}, which show consistency
between the old and new codes and with the injected signal parameters. The quantitative consistency between the codes can be seen in Table~\ref{tab:codeeval},
where is should be noted that the evidence ratio between the codes is larger than statistical uncertainty would expect, although the fractional difference between
the values ($\lesssim 1\%$) is still small enough to not greatly effect conclusions drawn from either values if used in model selection.

Similarly, using a signal with a coherent multi-detector signal-to-noise ratio of 8 (for the same parameters as above except with $h_0 = 3.5\ee{-23}$) when
simulated and injected into Gaussian noise for two detectors (H1
and L1) we see the posteriors shown in Figure~\ref{fig:simsignal_multi}, and consistency checks in Table~\ref{tab:codeeval}.

The notebooks for the above two tests can be found \href{https://github.com/mattpitkin/CW_nested_sampling_doc/blob/master/figures/codeeval/simulations/signal/SimulatedSignalTestsPaper.ipynb}{here} and 
\href{https://github.com/mattpitkin/CW_nested_sampling_doc/blob/master/figures/codeeval/simulations/signal_multidet/SimulatedSignalMultidetTestsPaper.ipynb}{here}.

A set of simulated signals with sky location and orbital parameters set to those of the Low-mass X-ray binary Sco X-1 were injected into
Gaussian noise for the study in \citet{2015PhRvD..92b3006M}. As a test of our code when recovering a signal from a source in a binary system
we have recovered one of these signals using approximately 1.4 days of simulated data. We have purposely performed the heterodyne data processing stage
with values of several of the phase parameters ($f_0$, $\alpha$, $T_0$, $a\sin{i}$ and $P_{\text{b}}$) set to {\it not} match the known simulated signal
values. Therefore, to recover the signal, in addition to the four \gw parameters we have had to allow the code to search over these extra parameters.
For these additional parameters Gaussian priors were used, with the prior means set to be the parameter values used for heterodyning (not those for
the actual signal), with standard devaitions wide enough to encompass the true signal values. For $h_0$ a Fermi-Dirac prior was used, whilst the other
angle parameters used priors covering their minimal allowed ranges. The recovered signal posterior distributions can be seen
in Figure~\ref{fig:scox1_inj}, which show that the true parameters are correctly recovered (the plot show recovered phase parameters as offsets from
the heterodyne parameter values, i.e.\ the centres of the Gaussian priors). It can be seen that for $f_0$ and $P_{\text{b}}$ the posteriors contain the
correct value, but are peaked well away from the value used for heterodyning (which would be a zero in the plots), showing the codes ability to explore
the prior space in these parameters. The one-dimensional marginal distribution for $\phi_0$ spans the full prior range, however, this is due to it being
highly correlated with $f_0$, and, although not visible on the plot, this very strong correlation is present in the two-dimensional posterior for these
parameters. Finally, for $\alpha$ and $T_0$ it can be seen that the posteriors just fill the prior ranges as the data give not additional information
about these parameters. For completeness, we find that the the signal was recovered with SNRs of $\sim 20$ in both H1 and L1 individually, with a coherent
SNR of $\sim 28$, and odds values for the multi-detector analysis of $\log{}_{10}\left(\mathcal{O}_{\text{S}/\text{N}}\right) = 152$
and $\log{}_{10}\left(\mathcal{O}_{\text{S}/\text{I}}\right) = 7.8$.

\subsection{Injections into real data}

\hl{Show results from an S5 hardware injection, an S6 software injections.}

Here we look recovering both hardware and software simulated signals injected into real LIGO data.

\subsection{Monte-Carlo studies}

In this section we will assess the outputs of \lppen using a range of simulated signals. We have created two sets of 2000 signal parameters to
injected simulated signals into Gaussian noise (with the same standard deviation of $1\ee{-22}$) for the H1 and L1 detectors. Both sets have been generated
with $\Phi_{22}^C$, $\psi$ and $\cos{\iota}$ drawn from uniform distributions over their minimal allowed ranges \citep[Table~1 of][]{2015MNRAS.453.4399P},
but the first set
calculates $h_0$ such that the coherent SNR (Equation~\ref{eq:snr}) is drawn uniformly between 0 and 20; the second set draws the $h_0$ uniformly between
0 and $3.25\ee{-22}$, such that the maximum coherent SNR (given the known noise level) is $\sim 55$. This latter set is used to evaluate the posterior
distributions as described below in \S\ref{sec:ppplots}. The source's sky locations are drawn uniformly from over the sky-sphere. For these analyses each
simulated data set is one solar day long, consisting of 1440 complex data points (generated as if heterodyned at precisely the known source phase evolution)
sampled once per minute. This data has been used to estimate the four parameters $h_0$, $\cos{\iota}$, $\Phi_{22}^C$ and $\psi$, with each being given
a flat prior: the angles and cosine of orientation angle over their minimal ranges, and $h_0$ between zero and $1\ee{-20}$ (well above the extent of the
bulk of the likelihood). When running \lppen on these, two parallel runs with 1024 live points where used, along with the default sampler proposals described
in \S\ref{sec:proposals}, and with the samples from both runs being combined to produce posterior samples and evidence estimates.

Another set of 2000 simulations has been created that again create simulated signals and add them to Gaussian noise for the H1 and L1 detectors. In this
case the total SNR shared between the detector has be drawn from a uniform distribution between 0 and 30, however, the signals have been purposely created
to be incoherent between detectors. The parameters $\Phi_{22}^C$, $\psi$ and $\cos{\iota}$ are drawn from their minimal allowed ranges, but are different for
the signals input into the two detectors, whilst the sky positions are also independent for both detectors. The signal rotational frequency and frequency
derivative are also offset between detectors, such that the offsets are drawn from Gaussian distributions of $1/2048$\,Hz and $1\ee{-10}$\,Hz$\,s^{-1}$
respectively. The length of the data is the same as in the previous case, but it is assumed that the data for both detectors was heterodyned using the
known phase evolution for the signal in the just H1 detector. The same prior and run settings as above were used. These simulations have been used for
assessing the odds for a coherent versus incoherent signal (see \S\ref{sec:odds}).

Finally, two sets of simulations of 500 signals have been created to assess the code when estimating additional phase parameters: in this case
the rotational frequency $f_0$, frequency derivative $\dot{f}_0$, and right ascension $\alpha$. Gaussian priors on these three parameters have been
used, with parameters given in Table~\ref{tab:gaussianpriors}. These Gaussian priors are used when generating signals and when estimating them
using \lppen, whilst the simulated data is created such that it was heterodyned using the means values. As above, one set of these simulations has amplitude 
calculated such that the coherent SNRs are drawn uniformly between 0 and 20, whilst the other has amplitude drawn uniformly between 0 and $3.25\ee{-22}$.
Again, the angles are drawn uniformly from their minimal ranges and sky positions are drawn uniformly over the sky-sphere.

\begin{table}[!hptb]
\caption{Gaussian prior parameters for rotational frequency, frequency derivative and right ascension for a set of simulated signals.
\label{tab:gaussianpriors}}
\begin{center}
\begin{tabular}{l | c c}
\hline
Parameter & mean & standard deviation \\                      
\hline
\hline
$f_0$ (Hz) & 100 & $5\ee{-5}$ \\
$\dot{f}_0$ & $-1\ee{-9}$ & $2\ee{-10}$ \\
$\alpha$ (rads) & $\pi$ & $0.0007272$ ($10^{\text{s}}$) \\
\hline
\end{tabular}
\end{center}
\end{table}

\subsubsection{Evaluating the posterior distributions}\label{sec:ppplots}

In \citet{2014PhRvD..89h4060S} a method was developed for evaluating whether sky location credible intervals for compact binary coalescence \gw signals, produced
from posterior distribution output by \lalinf codes \citep{2015PhRvD..91d2003V}, behaved self-consistently. The approach uses injected signal, with parameters
drawn from the prior distributions used when reconstructing them\footnote{In this case the amplitudes of the simulations are not drawn from the full prior range,
but are drawn from a flat distribution within that range, so are valid for this comparison.}, and sees if the credible intervals effectively behave as frequentist
confidence intervals, i.e.\ do 50\% of the known parameter values fall within some pre-specified definition (like the minimal credible region, or credible
region bounded by the lower extent to the prior) of the 50\% credible region. This can test whether the underlying \lalinf parameter sampling is working as
expected, or if there are biases to wider, or narrower, credible regions. The method, and the ``P-P plots'' it produces through evaluation over a range of credible
intervals, has now become commonly used to assess the \lalinf parameter estimation codes \citep{2015PhRvD..91d2003V} as a way of showing up any problems.

We have used this method here to evaluate our parameter posteriors produced from the simulations described above for which the amplitudes were drawn from
a uniform distribution. For each simulation we have calculated the minimal credible region from the one-dimensional marginalised posterior samples that the
injected value falls within (using a greedy-binning approach). The cumulative histogram of these values, for each parameter when using the simulations that
just estimate the four \gw parameters, can be seen in Figure~\ref{fig:pp_standard}. Also, shown for each parameter is a Kolmogorov-Smirnov test $p$-value
comparing our observed distribution to a uniform distribution, and a ``cloud'' of cumulative distributions that are random realisations of the expected
cumulative distribution. We see that, overall, our posteriors seem to produce posteriors that give credible intervals that behave as expected, without any
large deviations.

Similarly, we have done the same thing for the posteriors produced from the simulations that included searches over $f_0$, $\dot{f}_0$, and $\alpha$. These
P-P plots can be seen in Figure~\ref{fig:pp_extra}, where generally we see posterior credible intervals behaving as expected. However, we do find that the
initial phase parameter, $\Phi_{22}^C$, distributions are maybe broader than we would expect. This is most likely due to this parameter being highly
correlated with the frequency and not being well constrained, so the true distribution is not sampled well enough due to our finite number of posterior samples.
\hl{Maybe re-run with increased numbers of live points or parallel runs.}

\subsubsection{Evaluating the odds}\label{sec:evalodds}
