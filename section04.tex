\section{Evaluating the code}

In this section we will show how the posterior parameter distributions that we can generate from the output
of our nested sampling code compare to evaluating them on a grid, or with a previous implementation of the analysis code
\citep[\lppe used in e.g.][]{2014ApJ...785..119A} using a (generally inefficient) MCMC algorithm. We will show this for
simulated data containing purely Gaussian noise and containing fake signals added to Gaussian noise. We will also
show this for some hardware injections (signals directly induced in detector data) in real LIGO data from the fifth
science run (S5).

We will also directly compare some S5 results for searches for real signals from pulsars with those previously published
\citep{2010ApJ...713..671A}.

Finally we show how the odds comparing a signal model and a noise only model change with signal-to-noise ratio, along
with comparing coherent verses incoherent signals (for coherent simulated signal and incoherent simulations).

Add p-p plots!

\subsection{Simulated data}

In each of the tests below we will be running our code (\lppen) using the default proposal distributions discussed in \S\ref{sec:proposals}
and with the number of live points fixed to be 2048. We will also assume a source at a fixed sky position with right ascension of
\hms{00}{00}{00}{0} and declination \dms{00}{00}{00}{0}. The simulated data in all cases will be generated as the complex heterodyned
time series with 1440 points samples once per minute. When comparisons between the current code and previous codes are 
made we run the current code in a way that splits datasets into 30 point chunks, rather than using the algorithm described in \S{sec:splitting},
as this makes it consistent with the old code (\lppe). We also purely use the uniform prior ranges for the amplitude parameter to be
consistent between codes.

\subsubsection{Simulated noise}

An initial test of the code is whether the output posterior probability distributions match those produced when evaluating the
posterior over a fixed grid in the four-dimensional parameter space of $\vec{\theta} = \{h_0, \cos{\iota}, \psi, \phi_0\}$, when
using purely Gaussian noise.\footnote{It is worth noting that in many of these tests the polarisation angle prior covers $-\pi/4$ to $\pi/4$
rather than the 0 to $\pi/2$ range given in, e.g., \citet{2015MNRAS.453.4399P} due to this being the range required by the older
parameter estimation MCMC code. Although this range is degenerate to rotations over $\pi/2$, so spans the same range of waveform models}
A comparison of the marginalised posteriors for individual parameters, and pairs of parameters, are shown when using simulated data for
a single detectors (the LIGO Hanford detector, H1, and Livingston detector, L1), and the two detectors combined coherently, is shown
in Figure~\ref{}.
