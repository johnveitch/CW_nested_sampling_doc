\section{Evaluating the code}

In this section we will show how the posterior parameter distributions that we can generate from the output
of our nested sampling code compare to evaluating them on a grid, or with a previous implementation of the analysis code
\citep[\lppe used in e.g.][]{2014ApJ...785..119A} using a (generally inefficient) MCMC algorithm. We will show this for
simulated data containing purely Gaussian noise and containing fake signals added to Gaussian noise.

Finally, we show how the odds comparing a signal model and a noise only model change with signal-to-noise ratio, along
with comparing coherent verses incoherent signals (for coherent simulated signal and incoherent simulations).

Add p-p plots!

\subsection{Simulated data}

In each of the tests below we will be running our code (\lppen) using the default proposal distributions discussed in \S\ref{sec:proposals}
and with the number of live points fixed to be 2048. We will also assume a source at a fixed sky position and that the signal model is purely
that for the $l=m=2$ harmonic (Equation~\ref{eq:h2f}), and that we work in terms of the signal amplitude $h_0$ and signal phase
$\Phi_{22}^C$ (this is for consistency with the old code which uses $\Phi_{22}^C$ rather than $\phi_0$, which we have earlier defined as the
rotational phase). The simulated data in all cases 
will be generated as the complex heterodyned
time series sampled once per minute. When comparisons between the current code and previous codes are 
made we run the current code in a way that splits datasets into 30 point chunks, rather than using the algorithm described in \S\ref{sec:splitting},
as this makes it consistent with the old code (\lppe). We also purely use the uniform prior ranges for the amplitude parameter to be
consistent between codes.

\subsubsection{Simulated noise}\label{sec:simnoise}

An initial test of the code is whether the output posterior probability distributions match those produced when evaluating the
posterior over a fixed grid in the four-dimensional parameter space of $\vec{\theta} = \{h_0, \cos{\iota}, \psi, \phi_{22}^C\}$, when
using purely Gaussian noise.\footnote{It is worth noting that in many of these tests the polarisation angle prior covers $-\pi/4$ to $\pi/4$
rather than the 0 to $\pi/2$ range given in, e.g., \citet{2015MNRAS.453.4399P} due to this being the range required by the older
parameter estimation MCMC code. Although this range is degenerate to rotations over $\pi/2$, so spans the same range of waveform models.}
A comparison of the marginalised posteriors for individual parameters, and pairs of parameters, are shown when using simulated data lasting
ten days from a single detector (in this case assumed to be the LIGO Hanford detector, H1) in Figure~\ref{fig:simnoise_single}.

It can be seen that the output posteriors look qualititavely consistent. However, we can also quantify some aspects of consistency by
comparing the evidence output from the nested sampling code with that estimated from the grid-based method from \lppe, the upper limits
on $h_0$ produced by the codes, and performing Kolmogorov-Smirnov consistency tests between the nested sampling posterior samples
and MCMC posterior samples (giving a $p$-value for the null hypothesis that the samples are from the same distributions).
These are shown in Table~\ref{tab:codeeval} and show very good consistency between the codes, although it should be noted that
these are for one particular run and some statistical fluctuations in the exact values for different runs are to be expected (see e.g.\
the variations in evidence values in \S\ref{sec:proposaltesting}).

A {\tt jupyter} notebook with this test can be found \href{https://github.com/mattpitkin/CW_nested_sampling_doc/blob/master/figures/codeeval/simulations/noise/SimulatedNoiseTestsPaper.ipynb}{here}.

\begin{table*}[hptb]
\caption{Consistency tests between outputs of the new code, \lppen, and the old code, \lppe, when running on simulated data
and searching over the four parameters $\{h_0, \cos{\iota}, \psi, \Phi_{22}^C\}$.\label{tab:codeeval}}
\begin{center}
\begin{tabular}{l c c | c c c c}
\hline
\multirow{2}{*}{Simulation} & \multirow{2}{*}{$\ln{\left(\frac{Z_{\text{nested}}}{Z_{\text{grid}}}\right)}$} & \multirow{2}{*}{$\frac{(h_0^{95\%})_{\text{nested}}}{(h_0^{95\%})_{\text{grid}}}$} & 
\multicolumn{4}{c}{K-S $p$-value} \\ \cline{4-7}
 &  &  & $p(h_0)$ & $p(\Phi_C^{22})$ & $p(\cos{\iota})$ & $p(\psi)$ \\                      
\hline
\hline
Noise (single detector)  & $-0.07$ & $1.009$ & 0.404 & 0.026 & 0.010 & 0.703 \\
Noise (two detectors)    & $-0.05$ & $0.992$ & 0.141 & 0.564 & 0.538 & 0.493 \\
Signal (single detector) & 0.245   & $1.004$ & 0.237 & 0.291 & 0.125 & 0.175 \\
Signal (two detectors)   & 0.240   & 0.991   & 0.722 & 0.052 & 0.067 & 0.032 \\
\hline
\end{tabular}
\end{center}
\end{table*}

We see a very similar situation, in terms of agreement between the codes, when running on simulated data assumed to be from two detectors (the LIGO
H1 and L1 sites) as shown in Figure~\ref{fig:simnoise_single} and the second line of Table~\ref{tab:codeeval}. A {\tt jupyter} notebook with this test
can be found \href{https://github.com/mattpitkin/CW_nested_sampling_doc/blob/master/figures/codeeval/simulations/noise_multidet/SimulatedNoiseTestsMultidetPaper.ipynb}{here}.

The above tests using simulated noise and with searches covering the four standard unknown \gw parameter show that the code appears to
be working as expected, and is in agreement with our previously used code. However, more generally we can perform parameter estimation
over a larger number of signal parameters, and can again check for consistency with the previous code. We produce one day of Gaussian
noise for a single detector and search over the four \gw parameter with the same priors as before, but then also use a multi-variate Gaussian prior (\S\ref{sec:gaussianprior})
over the {\it phase} parameters: rotational frequency and first frequency derivative, and binary parameters of binary period, time of periastron, angle of
periastron and its first derivative, projected semi-major axis, and eccentricity ($\{f,\dot{f},P_{\text{b}}, T_0, \omega_0, \dot{\omega}_0, a\sin{i}, e\}$).
This gives a total of 12 parameters in the search. The multi-variate prior is defined such that all parameters are uncorrelated except, in this case,
the parameter pairs $[T_0, \omega_0]$ and $[P_{\text{b}}, \dot{\omega}_0]$ for which we use a very high correlation of $0.9999$ (we do not set them to
be fully correlated due to numerical issues inverting such matrices).

The one-and-two dimensional posteriors output for this case when run with both \lppe and \lppen can be seen in Figure~\ref{fig:noise_multiparam},
which also overlays the marginalised priors on top. It can be seen qualititavely that the codes are consistent\footnote{It is worth mentioning that in performing
these tests a bug was discovered in \lppe in which Equation~\ref{eq:deltaphi} was being applied with the wrong sign, leading to parameter estimates
having the wrong sign. However, this bug was fixed for the example shown here.} and for the {\it phase} parameter priors are recovered, except for
$\dot{f}$, which has some structure due to the specifics of the noise realisation. The Kolmogorov-Smirnov test $p$-values testing the null hypothesis
that the posterior samples from each code are drawn from the same distribution are shown in Table~\ref{tab:noisemultiks}, and suggest that the
distributions are consistent.

\begin{table*}[hptb]
\caption{Kolmogorov-Smirnov test $p$-values testing the null hypothesis that the samples output by \lppen and \lppe, when running on simulated
noise data and searching over the twelve parameters $\{h_0, \cos{\iota}, \psi, \Phi_{22}^C, f,\dot{f},P_{\text{b}}, T_0, \omega_0, \dot{\omega}_0, a\sin{i}, e\}$,
are drawn from the same distributions.\label{tab:noisemultiks}}
\begin{center}
\begin{tabular}{l | c c c c c c c c c c c c}
\hline
Parameter & $h_0$ & $\Phi_C^{22}$ & $\cos{\iota}$ & $\psi$ & $f$ & $\dot{f}$ & $P_{\text{b}}$ & $T_0$ & $\omega_0$ & $\dot{\omega}_0$ & $a\sin{i}$ & $e$ \\                      
\hline
\hline
$p$-value  & 0.13 & 0.04 & 0.78 & 0.23 & 0.44 & 0.26 & 0.85 & 0.28 & 0.29 & 0.86 & 0.52 & 0.28 \\
\hline
\end{tabular}
\end{center}
\end{table*}

\subsubsection{Simulated signal}\label{sec:simsignal}

The next test is checking how the codes compare when there is a signal present in the data. When generating the simulated signals we initially
assume that the signal's phase evolution is perfectly known and has been removed via the heterodyne described in \S\ref{sec:model}, and therefore
$\Delta\phi_2 = 0$ from Equation~\ref{eq:deltaphi}. As in \S\ref{sec:simnoise} then will search over the four-dimensional parameter space of
$\vec{\theta} = \{h_0, \cos{\iota}, \psi, \Phi_{22}^C\}$.

We create a simulated signal in the H1 detector spanning ten days of data with parameters ($h_0 = 6.2\ee{-23}$, $\Phi_{22}^C = 2.4$\,rad, $\cos{\iota} = 0.3$
and $\psi = 0.1$\,rad) that produce a signal-to-noise ratio of 8 when injected into
Gaussian noise. The posterior probability distributions estimated for the parameters (expressed as offsets from the assumed heterodyne parameter values) are
shown in Figure~\ref{fig:simsignal_single}, which show consistency
between the old and new codes and with the injected signal parameters. The quantitative consistency between the codes can be seen in Table~\ref{tab:codeeval},
where is should be noted that the evidence ratio between the codes is larger than statistical uncertainty would expect, although the fractional difference between
the values ($\lesssim 1\%$) is still small enough to not greatly effect conclusions drawn from either values if used in model selection.

Similarly, using a signal with a coherent multi-detector signal-to-noise ratio of 8 (for the same parameters as above except with $h_0 = 3.5\ee{-23}$) when
simulated and injected into Gaussian noise for two detectors (H1
and L1) we see the posteriors shown in Figure~\ref{fig:simsignal_multi}, and consistency checks in Table~\ref{tab:codeeval}.

The notebooks for the above two tests can be found \href{https://github.com/mattpitkin/CW_nested_sampling_doc/blob/master/figures/codeeval/simulations/signal/SimulatedSignalTestsPaper.ipynb}{here} and 
\href{https://github.com/mattpitkin/CW_nested_sampling_doc/blob/master/figures/codeeval/simulations/signal_multidet/SimulatedSignalMultidetTestsPaper.ipynb}{here}.
