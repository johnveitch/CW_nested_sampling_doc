\section{Nested sampling setup}

Here we will briefly describe some of the core functions used for the Nested Sampling algorithm within our code. We will
not describe nested sampling itself, over-and-above the very brief summary given in \S{sec:general}, but instead refer
the reader to \citet{Veitch:2010}. We also refer the reader to \citep{2015PhRvD..91d2003V} for a more thorough description
of some of the concepts we will discuss below.

One main point about the nested sampling algorithm is that it starts by randomly drawing a certain number of samples (the number of
which we call the number of live points, $N_{\rm live}$, which will be the constant number of ``active'', or ``live'', samples),  
where a sample is a vector containing a value for each parameter $\vec{\theta}$, from the prior distribution (see \S\ref{sec:priorfuncs}).
The samples must be independent draws. For each of these samples the likeihood will be calculated and the sample $\vec{\theta}_{\rm min}$
corresponding to the minimum likelihood, $L_{\rm min} = p(\mathbf{B}|\vec{\theta}_{\rm min})$, will be noted. As the algorithm
progresses the number of live points stays the
same, but at each iteration a new sample will be added in place of the one with the minimum likelihood, and therefore the
minimum likelihood will continuously be updated to be the minimum value for the current set of points.

\subsection{Proposal functions}

The nested sampling algorithm requires a way to draw new samples from the parameter space that is being explored. The
particular requirement for nested sampling is that any new sample is an independent draw from a constrained version
of prior probabilty function for each parameter. The constraint is that the new sample's likelihood must be greater
than the current value of $L_{\rm min}$, e.g. for a single parameter $x$, the probability of drawing a point at $x_i$ 
would be
\begin{equation}
 p(x=x_i|I)_{\rm constrained} = \begin{cases}
             p(x=x_i|I) & \text{if~} p(\mathbf{B}|x_i) > L_{\rm min}, \\
             0 & \text{otherwise},
            \end{cases}
\end{equation}
where $p(x|I)$ is the prior on $x$.

There are various methods to sample from the constrained prior, the simplest of which is just to samples from the full
prior and only accept a sample if it meets the likelihood constraint. However, such a simple method soon becomes very
inefficient if the likelihood gets more and more tightly concentracted within the prior volume (which can happen rapidly
for high-dimensional problems). One popular nested sampling method is MultiNest \citep{2009MNRAS.398.1601F}, which performs
the sampling by calculating a set of ellipsoids bounding the live points and drawing points uniformly from within them.
Various other approaches exist such as Diffusive Nested Sampling \citet{Brewer2011,2016arXiv160603757B} and POLYCHORD \citet{2015MNRAS.450L..61H}.

The version of nested sampling used in the \lalinf library, as used by our code, makes use of a Markov chain Monte Carlo (MCMC) approach to
drawing new samples \citep{Veitch:2010}. The MCMC attempts to sample from the constrained prior distributions in an efficient manner, with
the final sample of the chain being a statistically independent samples from the rest of the live points. To sample most efficiently
the MCMC requires a proposal distribution that closely resembles the underlying constrain prior that it is trying to sample from.
\lalinf has a variety of potential proposal distributions \citep{2015PhRvD..91d2003V}, but our code currently allows five different
proposals, that can be used in different proportions:
\begin{description}
 \item[Ensemble walk] This is one of the affine invarient ensemble proposals described by \citet{GoodmanWeare}, which adopt the
 scale and shape of the current set of samples. The hard-coded setup in \lalinf is that three samples randomly drawn from a cache
 of samples are used to define the walker samples. This is the main default proposal of our code, and is used as the proposal
 75\% of the time.
 \item[Ensemble stretch] This is another affine invarient ensemble proposal described by \citet{GoodmanWeare}, and makes use of
 two samples randomly drawn from the sample cache. By default this proposal is not used at all.
 \item[Differential evolution] This is similar to the stretch proposal in that it uses two samples randomly drawn from the sample
 cache. By default this proposal is not used at all.
 \item[Uniform proposal] This proposal just draws points from their full prior for parameters that have a uniform prior specified.
 This proposal ise used 25\% of the time by default.
 \item[Frequency jump] This proposal tries jumping between Fourier frequency bins if frequency is one of the required parameters.
 This proposal has not been tested and, by default, is not used at all.
\end{description}
The cache of samples mentioned above, and used by the nested sampling algorithm, is the set of live points, but this cache is
only updated on iterations of the algorithm that are a factor of 10\% of the number of live points.

The length of each MCMC run can be chosen by the user, but by default it is automatically set within the code. To do this,
at the same rate as the cache is updated, the autocorrelation length of the parameters is calculated. A random sample from the current
live points in chosen, and evolved via the MCMC, and the autocorrelation length of each parameter is calculated, $\vec{{\rm acl}}$. The number
of MCMC iterations, $n_{\rm MCMC}$, is then chosen as $n_{\rm MCMC} = {\rm min}({\rm max}(\vec{{\rm acl}}), 5000)$, where 5000 is
a hardcoded maximum length.

\subsubsection{Testing proposals}

We would like to test whether the above proposals lead to accurate calculations of the evidence integral as given in Equation~\ref{eq:evidence}.
It is especially interesting to do this over ranges of prior volumes that are initially similar to the posterior volume, but then diverge to be
much larger than the posterior volume (or in other words, as the information gain, or Kullback–Leibler divergence, increases). To do this
we have created a simple dataset consisting of 1440 complex samples drawn from a Gaussian distribution (and sampled at 1/60\,Hz) and, assuming
a signal model of the form given by Equation~\ref{eq:h2f} (but using the $h_0$ amplitude convention), calculated the signal model evidence
by numerically integrating over $h_0$, whilst holding the other parameters fixed at zero, for a uniform prior (where the prior's lower range was zero
and upper range was set to a value at which the likelihood was very small). We have then run our code multiple times over using a range of 
upper values on a uniform prior on $h_0$ (starting at the same value as used for the numerical calculation) using different combinations of
.

