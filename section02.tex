\section{Core functions}

Here we will describe the core parts of the code defining the signal model and the probability functions used
for the inference. These assume that the data consists of calibrated narrow-band complex heteroyned time
series'. These time series data streams can be from different detectors, and/or different heterodyne
frequencies. For example you could have a pair times series from both the LIGO Hanford (H1) and LIGO
Livingston (L1) detectors, with one produced with a heterodyne at twice the rotation frequency of a known
pulsar and the other produced with a heterodyne at the rotation frequency.

\subsection{The signal model}\label{sec:model}

Our code assumes that the rotational phase evolution of a pulsar is defined by the Taylor expansion in phase
\begin{equation}
\phi(t) = 2\pi\left(fT + \frac{1}{2}\dot{f}T^2 + \frac{1}{6}\ddot{f}T^3  \ldots\right)
\end{equation}
where $T$ is the time corrected to an inertial reference frame (the solar system barycentre
for isolated pulsars, or the binary system barycentre for pulsars in binaries), and the $f$'s give
the rotation frequency and its time derivatives. The value of $T = (t+\tau(t)-t_0)$ where $t$ is the
time at a detector, $\tau(t)$ is the time dependent correction to the inertial frame and $t_0$ is the epoch.
In practice the code can currently accept frequency derivatives up to the ninth order. We assume the
calibrated detector data $d(t)$ has been heterodyned such that
\begin{equation}
B'(t) = d(t)e^{-i\Phi_{i,{\textrm het}}(t)},
\end{equation}
where $\Phi_{i,{\textrm het}}(t)$ is the phase evolution for a given data stream $i$, and we produce $B(t)$ by
low-pass filtering and averaging values of $B'(t)$.

Under the standard assumption that the general theory of relativity (GR) is correct the code uses the form of
the signal model defined in \citet{2015arXiv150105832J}, which, when heterodyned and asumming low-pass
filtering, gives a signal at a pulsar's rotation frequency (where $\Phi_{1,{\rm het}}(t) = \phi(t)$) of
\begin{widetext}
\begin{equation}\label{eq:hf}
h_f(t) =  e^{i\Delta\phi_1(t)}\left(-\frac{C_{21}}{4}F_{+}(\psi,t)\sin{\iota}\cos{\iota}e^{i\Phi_{21}^C} +
i\frac{C_{21}}{4}F_{\times}(\psi,t)\sin{\iota}e^{i\Phi_{21}^C} \right)
\end{equation}
and at twice the pulsar's rotation frequency (where $\Phi_{2,{\rm het}}(t) = 2\phi(t)$) of
\begin{equation}\label{eq:h2f}
h_{2f}(t) =  e^{i\Delta\phi_2(t)}\left(-\frac{C_{22}}{2}F_{+}(\psi,t)[1+\cos{}^2\iota]e^{i\Phi_{22}^C} +
iC_{22}F_{\times}(\psi,t)\cos{\iota}e^{i\Phi_{22}^C} \right).
\end{equation}
\end{widetext}
The $F_{+}$ and $F_{\times}$ values are the detector dependent antenna patterns, which are a function of the
detector position, source sky position and source polarisation angle $\psi$. The $C_{21}$, $C_{22}$,
$\Phi_{21}^C$ and $\Phi_{22}^C$ values are convenient ways of representing the waveform in terms of an
amplitude and phase of the signal for the $l=2$, $m=1$ harmonic and $l=m=2$ harmonic respectively. The
$\Delta\phi(t)$ values represent any time dependent phase deviation between the phase used in the heterodyne
and the true signal phase (which does not necessarily have to precisely follow the true rotational phase), so
$\Delta\phi_1(t) = (\phi_{1,{\rm true}}(t)-\Phi_{1,{\rm het}}(t))$ and $\Delta\phi_2(t) = (\phi_{2,{\rm
true}}(t)-\Phi_{2,{\rm het}}(t))$.

To calculate the $\Delta\phi$ values using up to the $(n-1)^{\rm th}$ frequency derivative, and try to avoid numerical overflow issues when dealing with large
phases, the following equation is used
\begin{widetext}
\begin{equation}\label{eq:deltaphi}
\Delta\phi_j(t) = 2\pi \sum_{k=1}^n \left( \frac{\left(f^{(k-1)}_{j,{\rm true}} - f^{(k-1)}_{j,{\rm het}}\right)}{k!}(t+\delta t_{\rm het})^k + \frac{f^{(k-1)}_{j,{\rm true}}}{k!} \sum_{i=0}^{i<k} \left(\begin{array}{c}k \\ i\end{array} \right) (\delta t_{\rm true}-\delta t_{\rm het})^{k-i} (t+\delta t_{\rm het})^i \right),
\end{equation}
\end{widetext}
where $f^{(n)}$ is the $n^{\rm th}$ frequency derivative, and $\delta t$ is the combination of any solar system barycentring and binary system
barycentring time delays.

By default the code assumes emission just from the $l=m=2$ mode, i.e.\ there is only a signal at twice the
rotation frequency. In this case $C_{22}$ and $\Phi_{22}^C$ can be related to the more familiar physical
$h_0$ and $\phi_0$ values via $h_0 = -2C_{22}$ \citep[where the minus sign maintains consistency of equation~\ref{eq:h2f} with the form given in][]{1998PhRvD..58f3001J} and pulsar rotational 
phase $\phi_0 = \Phi_{22}^C/2$. For the more general case the
relations between the waveform amplitude and phase parameters and physical source parameters are given in
\citet{2015arXiv150105832J} and \citet{2015MNRAS.453.4399P}. In general, for previous searches we have often assumed that we track the true
signal phase perfectly with the heterodyne and as such $\Delta\phi_i(t) = 0$. In such cases the only time
varying components of the signal are the antenna pattern functions, which allows great speed increases in the
signal generation and likelihood calculations (see \S\ref{sec:fastlike}).

\subsection{The likelihood functions}\label{sec:likelihood}

Our code can make use of two different likelihood functions. The default likelihood function is a
Student's {\it t}-likelihood function in which it is assumed that the standard deviation of the noise in the
data is unknown, and can therefore be marginalised over. A Gaussian likelihood function can also be used, for
which the code can either take in estimates of the noise standard deviation at each data point, or calculates
these internally based on stationary stretches of data. For the Student's {\it t}-likelihood function, and if
calculating noise standard deviations for the Gaussian likelihood function internally, the code needs to break
up the data into chunks that have (roughly) the same distribution. The method for doing this is given in
\S\ref{sec:splitting}.

\subsubsection{Student's {\it t}-likelihood}\label{sec:stlikelihood}

A full derivative of the Student's {\it t}-likelihood function \citep[see e.g.][]{2005PhRvD..72j2002D} is given
in Appendix~\ref{app:likelihood}, but the final form of the joint likelihood (and its natural logarithm,
which is actually used within the code to maintain precision) for multiple detectors and data streams is
given by
\begin{widetext}
\begin{align}\label{eq:stlikelihood}
p(\mathbf{B}|\vec{\theta}) &= \prod_{i=1}^{N_{\rm dets}} \prod_{j=1}^{N_{\rm s}} \prod_{k=1}^{M_{i,j}}
\frac{(m_{i,j,k}-1)!}{2\pi^{m_{i,j,k}}}
\left(\sum_{n=n_{i,j,0}}^{n_{i,j,0}+(m_{i,j,k}-1)} |B_{i,j,n}-y(\vec{\theta})_{i,j,n}|^2\right)^{-m_{i,j,k}},
\nonumber \\
\ln{p(\mathbf{B}|\vec{\theta})} &= \sum_{i=1}^{N_{\rm dets}} \sum_{j=1}^{N_{\rm s}}
\sum_{k=1}^{M_{i,j}} \left( \mathcal{A}_{i,j,k} - m_{i,j,k}\ln{
\left\{\sum_{n=n_{i,j,0}}^{n_{i,j,0}+(m_{i,j,k}-1)} |B_{i,j,n}-y(\vec{\theta})_{i,j,n}|^2\right\}}
\right),
\end{align}
\end{widetext}
where $N_{\rm dets}$ is the number of detectors used, $N_{\rm s}$ is the number of data streams (e.g.\
heterodyned data from both the rotation frequency and twice the rotation frequency) per detector, $M_{i,j}$ is
the total number of independent data chunks for detector $i$ and data stream $j$ with lengths $m_{i,j,k}$ and
$n_{i,j,0} = \sum_{l=1}^{k} 1+m_{i,j,l-1}$ (with $m_{i,j,0} = 0$) being the index of the first data point in
each chunk. The model $y(\vec{\theta})$ is that given by Eqns.~\ref{eq:hf} and/or \ref{eq:h2f}
depending on which data streams are being analysed. For notational convenience we have made the substitution
$\mathcal{A}_{i,j,k} = \ln{\left([m_{i,j,k}-1]!\right)} - \ln{2} - m_{i,j,k}\ln{\pi}$.

\subsubsection{Gaussian likelihood}\label{sec:glikelihood}

The Gaussian likelihood, and its natural logarithm, are similarly given by
\begin{widetext}
\begin{align}\label{eq:gausslikelihood}
p(\mathbf{B}|\vec{\theta}) &= \prod_{i=1}^{N_{\rm dets}} \prod_{j=1}^{N_{\rm s}} \prod_{k=1}^{L_{i,j}}
\frac{1}{2\pi\sigma_{i,j,k}^2}\exp{\left(-\frac{|B_{i,j,k}-y(\vec{\theta})_{i,j,k}|^2}{2\sigma_{i,j,k}^2}
\right)}, \nonumber \\
\ln{p(\mathbf{B}|\vec{\theta})} &= \sum_{i=1}^{N_{\rm dets}} \sum_{j=1}^{N_{\rm s}}
\sum_{k=1}^{L_{i,j}} \left(\mathcal{B}_{i,j,k} -
\left[\frac{|B_{i,j,k}-y(\vec{\theta})_{i,j,k}|^2}{2\sigma_{i,j,k}^2 } \right] \right)
\end{align}
\end{widetext}
where $L_{i,j}$ is the length of each dataset and $\mathcal{B}_{i,j,k} = -\ln{(2\pi\sigma_{i,j,k}^2)}$. Note
that the normal square root on the normalisation factor is not there because the exponential is already the
product of the real and imaginary data components.

\subsubsection{The null likelihood}

As we will most often want to perform model comparison for the signal model against the data just containing
noise (the null hypothesis in this case) we can define the null likelihoods for both of the above likelihoods
by setting $y(\vec{\theta}) = 0$, so that we have
\begin{align}\label{eq:nulllike}
\ln{p(\mathbf{B}|y=0)} &= \sum_{i=1}^{N_{\rm dets}} \sum_{j=1}^{N_{\rm s}}
\sum_{k=1}^{M_{i,j}} \Bigg( \mathcal{A}_{i,j,k} - \nonumber \\
&m_{i,j,k}\ln{
\left\{\sum_{n=n_{i,j,0}}^{n_{i,j,0}+(m_{i,j,k}-1)} |B_{i,j,n}|^2\right\}}
\Bigg),
\end{align}
and
\begin{equation}
\ln{p(\mathbf{B}|y=0)} = \sum_{i=1}^{N_{\rm dets}} \sum_{j=1}^{N_{\rm s}}
\sum_{k=1}^{L_{i,j}} \left(\mathcal{B}_{i,j,k} -
\left[\frac{|B_{i,j,k}|^2}{2\sigma_{i,j,k}^2 } \right] \right)
\end{equation}
for the Students-{\it t} and Gaussian likelihoods respectively.

If we were only interested in comparing models calculated using equivalent likelihood functions we could in
general ignore the factors that do not depend on the data or the model, as they would cancel in any odds
ratio. But, in this code we keep them for cases when such a comparison is not performed, e.g.\ if we
want to compared the joint multi-detector likelihood for a signal with the incoherent product of likelihoods
from each detector then we would need these factors to still be present.

\subsubsection{Fast likelihood evaluations}\label{sec:fastlike}

In cases when the only time varying components of the model are the antenna pattern functions (i.e.\ when 
$\Delta \phi_j(t)$ in equation~\ref{eq:hf} or \ref{eq:h2f} is zero) the likelihood
evaluation can be greatly sped-up by pre-calculated the components in the internal summations. For a given sky
position and detector the antenna patterns can be defined by \citep{1998PhRvD..58f3001J}
\begin{align}
F_+(t,\psi) &= \sin{\zeta}\left[a(t)\cos{2\psi} + b(t)\sin{2\psi}\right], \nonumber \\
F_{\times}(t,\psi) &= \sin{\zeta}\left[b(t)\cos{2\psi} - a(t)\sin{2\psi}\right],
\end{align}
where $\psi$ is the \gw polarisation angle, $\zeta$ is the known angle between the detector arms (generally
$90^{\circ}$), and $a(t)$ and $b(t)$ are the time dependent functions for a given detector position and
source sky location that vary over a sidereal day. These functions can be precomputed at a set of times over
a sidereal day and used, via look-up table interpolation, to give the value at any other time (our code
defaults to calculate $a(t)$ and $b(t)$ at 2880 points over a sidereal day). If, for example, we take the
imaginary part of the cross polarisation component of a signal model, then the summation in the likelihood for a
single detector (with $\zeta = 90^{\circ}$), single data stream and single chunk of length $n$ will be given
by
\begin{widetext}
\begin{align}
S =& \sum_{i=1}^n (\Im{(B_i)}-\left[a(t_i)\cos{2\psi} -
b(t_i)\sin{2\psi}\right]\mathcal{C})^2 \nonumber \\
 =& \sum_{i=1}^n \Im{(B_i)}^2 + \mathcal{C}^2\sum_{i=1}^n \left[a(t_i)\cos{2\psi} -
b(t_i)\sin{2\psi}\right]^2 - 2\mathcal{C}\sum_{i=1}^n  \Im{(B_i)}\left[a(t_i)\cos{2\psi} -
b(t_i)\sin{2\psi}\right], \nonumber \\
=& \sum_{i=1}^n \Im{(B_i)}^2 + \mathcal{C}^2\cos{}^2{2\psi}\sum_{i=1}^n a(t_i)^2 +
\mathcal{C}^2\sin{}^2{2\psi}\sum_{i=1}^n b(t_i)^2 - 2\mathcal{C}^2\cos{2\psi}\sin{2\psi}\sum_{i=1}^n
a(t_i)b(t_i) - \nonumber \\
& 2\mathcal{C}\cos{2\psi} \sum_{i=1}^n \Im{(B_i)}a(t_i) + 2\mathcal{C}\sin{2\psi} \sum_{i=1}^n
\Im{(B_i)}b(t_i),
\end{align}
\end{widetext}
where $\mathcal{C}$ represents the imaginary component of waveform amplitude for a given set of parameters.
It can be seen that all the summation terms can be pre-computed. If using the Gaussian likelihood this
pre-computation of the summation terms can also be done, but with the substitutions $B_i \rightarrow
B_i/\sigma_i$, $a(t_i) \rightarrow a(t_i)/\sigma_i$ and $b(t_i) \rightarrow b(t_i)/\sigma_i$. For the model,
assuming GR emission just from the $l=m=2$ mode, it is just the four different $\mathcal{C}$ terms that need
to be calculated during the sampling process within the code.

In the case where we want to search over parameters that mean that $\Delta\phi_i(t) \ne 0$ in the model then
we can not perform this pre-summing. However, {\it reduced order quadrature} methods
\citep[e.g.][]{2014PhRvX...4c1006F, 2015PhRvL.114g1104C} may help in these cases. Such a method is implemented in our
code, but will be discussed in a separate paper.

\subsection{The prior functions}\label{sec:priorfuncs}

An important part of any Bayesian inference method is the choice of parameter prior probability functions. Here we describe the prior
functions allowed in the code for any of the required parameters.

Certain parameters are physically not allowed to take negative values, so it is hardcoded that the following parameters
have zero probability below zero: gravitational wave amplitude, $h_0$; $l=m=2$ mass quadrupole moment, $Q_{22}$; the pulsar
distance; the pulsar parallax; the speed of gravitational waves; the projected semi-major axis of a binary orbit; the total
binary system mass; and, the companion mass in a binary system.

There are also other hardcoded priors for waveform amplitudes in two particular cases when searching for emission at both once
and twice the rotation frequency: if searching for a signal from a biaxial star then the two waveform amplitudes, $C_{21}$ and $C_{22}$,
must either both be positive or both be negative; if using the source model then the two amplitudes $I_{21}$ and $I_{31}$, must
have $I_{31} \geqslant I_{21}$.

\subsubsection{Uniform prior}

A parameter can be given a uniform (or flat, or top-hat) prior, in which the probabilty is constant within a given
range and zero outside that range, i.e.
\begin{equation}
p(x|I) = \begin{cases}
             C & \text{if } x_{\rm min} < x < x_{\rm max}, \\
             0 & \text{otherwise},
            \end{cases}
\end{equation}
where for parameter $x$ the upper and lower bounds on the parameter values and $x_{\rm min}$ and $x_{\rm max}$. This
prior can be normalised by setting $C = 1/(x_{\rm max}-x_{\rm min})$, although within nested sampling the normalisation
is not specifically required. The prior requires a lower and upper bound to allow it to be normalisable and to enable
an initial set of samples to be drawn from it. However, it should be noted that the choice of range will have an
effect on evidences that are calculated.

A histogram of a set of samples produced by the code when drawing from a uniform prior distribution are shown in Figure~\ref{fig:prioruniform},
along with the true prior function.

Some parameters for which the uniform prior is often considered appropriate are angle parameters, such as the $\Phi_{22}^C$ and $\psi$
parameters in Equation~\ref{eq:h2f}, which can be restricted to a specific non-degenerate range \citep[see, e.g., Table~1 in][]{2015MNRAS.453.4399P}. 
In previous gravitational wave pulsar searches, such as \citet{2010ApJ...713..671A,2014ApJ...785..119A} a uniform prior was often used for
the gravitational wave amplitude parameter $h_0$. For those searches the prior upper bound was based on inspection of the data to find a
value that was very large compared to the sensitivity suggested by the data. A uniform prior was used, as opposed to the more
uninformative choice for such as scale parameter of the log-uniform prior (see \S\ref{sec:loguniform}), due to the fact that the main
result was in producing upper limits, and wanting these to be primarily based on the data (i.e. the likelihood) rather than being reduced
by the prior.

\subsubsection{Gaussian prior}

A parameter can be given a prior with a Gaussian distribution, defined by its mean $\mu_x$ and standard deviation $\sigma_x$, i.e.
\begin{equation}
 p(x|I) = \frac{1}{\sqrt{2\pi\sigma_x^2}}\exp{\left(-\frac{(x-\mu_x)^2}{2\sigma_x^2}\right)}.
\end{equation}

A histogram of a set of samples produced by the code when drawing from a Gaussian prior distribution are shown in Figure~\ref{fig:priorgaussian},
along with the true prior function.

\subsubsection{Gaussian Mixture Model prior}

A parameter, or pair of parameters, can be given a prior distribution composed of a superposition of Gaussian distributions (a
Gaussian Mixture Model), with each component defined by a set of means for each parameter, a covariance matrix, and a weight
describing the relative probability concentrated in it. For an $n$ component mixture on a set of parameters $\vec{x}$ the prior
function will be
\begin{equation}
 p(\vec{x}|I) = \sum_{i=1}^n w_i(2\pi)^{n/2}|C_i|^{-1/2}\exp{\left(-\frac{1}{2}{\Delta\vec{x}_i}'C_i^{-1}\Delta\vec{x}_i\right)},
\end{equation}
where, for component $i$, $w_i$ is the weight, $\Delta\vec{x_i}= \vec{x}-\vec{{\mu_x}_i}$, $\vec{{\mu_x}_i}$ are the means, and
$C_i$ is the covariance matrix.

Such a model can be used, for example, if there is a bimodal Gaussian prior on a particular parameter, as is the case for the
inclination angle $\iota$ if it is calculated from fits to pulsar wind nebulae \citep[see, e.g., Appendix B in][]{2017arXiv170107709T}. In such a case the Gaussian Mixture Model can define
two Gaussian distributions of equal standard deviations and weights. A more complex use of this prior would be for creating a smooth
probability distribution using samples drawn from it, i.e. taking posterior samples from a search for a particular pulsar, and
creating a smooth model of those samples (for, say, the two parameters $h_0$ and $cos{\iota}$) to be input as a prior when looking
at future data for the same source.

Two examples of histrograms of samples drawn from one-dimensional Gaussian Mixture Models with two and three modes respectively
are shown in Figure~\ref{fig:priorgmm}.

\subsubsection{Log-Uniform prior}\label{sec:loguniform}

A parameter can be given a prior distribution that is uniform in the logarithm of the parameter, i.e.
\begin{equation}
 p(x|I) = \begin{cases}
             \left(\ln{x_{\rm max}}-\ln{x_{\rm min}}\right)^{-1}\frac{1}{x} & \text{if } x_{\rm min} < x < x_{\rm max}, \\
             0 & \text{otherwise}.
            \end{cases}
\end{equation}
This prior requires minimum and maximum values to be specified to make it normalisable and allow samples to be drawn from the
distribution. It should be noted that the choice of range, and in particular the lower bound, can have a significant effect on
the calculated evidence.

A histogram of a set of samples produced by the code when drawing from a log-uniform prior distribution, with a range between
$10^{-3}$ and 1 are shown in Figure~\ref{fig:priorloguniform},
along with the true prior function.

\subsubsection{Fermi-Dirac prior}

A Fermi-Dirac distribution prior function was inspired by that used in \citet{Middleton_2015}. The prior has a sigmoid, or logistic,
shape, although starts at large values and slopes downwards and is restricted to positive values. The form of the function, for
parameter $x$, is
\begin{equation}\label{eq:fermidirac}
 p(x|\sigma, \mu, I) = \begin{cases}\frac{1}{\sigma\ln{\left(1+e^{\mu/\sigma} \right)}}\left(e^{(x-\mu)/\sigma} + 1\right)^{-1} & \text{if } x \geqslant 0, \\
                        0 & \text{otherwise},
                       \end{cases}
\end{equation}
and is defined by two parameters $\sigma$ and $\mu$. $\mu$ defines the point at which the distribution falls to 50\% of
its maximum value, whilst $\sigma$ defines the range over which the attenuation happens. We also make use of a parameter
$r = \mu/\sigma$. The band over which the probability falls from 97.5\% to 2.5\% of its maximum value is given by $\mu \pm Z\mu/2r$, where
$Z\approx7.33$. Therefore, if we want, for example, the distribution to have this particular fall of over a range that is 10\% of $\mu$
we would have $Z/2r = 0.1$ and $\sigma = 0.027\mu$.

Three different examples of Fermi-Dirac priors, and samples drawn from them by the code, are shown in Figure~\ref{fig:priorfermidirac},
for a selection of $\mu$ and $\sigma$ values.

The sampling from the prior uses inverse transform sampling. The cumulative distribution function (CDF) of Equation~\ref{eq:fermidirac}
is
\begin{widetext}
\iflatexml
\begin{equation}
\else
\begin{align}
\fi
 C(X_0) &= \int_0^{X_0} \frac{1}{\sigma\ln{\left(1+e^{r} \right)}}\left(e^{(x-\mu)/\sigma} + 1\right)^{-1} {\rm d}x, \nonumber \\
 &= \frac{1}{\ln{\left(1+e^{r}\right)}}\left[\frac{X_0}{\sigma} + \ln{\left(1+e^{-r} \right)} - \ln{\left( 1+e^{(X_0-\mu)/\sigma}\right)} \right],
\iflatexml
\end{equation}
\else
\end{align}
\fi
which can be inverted to give
\begin{equation}
X_0 = -\sigma \ln{}\left(-e^{-r} + \left(1+e^{r} \right)^{-C(X_0)} + e^{1-r}\left(1+e^{r} \right)^{-C(X_0)} \right).
\end{equation}
\end{widetext}
Thus, drawing CDF samples uniformly between 0 and 1, and inverting via the above equation gives values drawn from Equation~\ref{eq:fermidirac}.

This prior can use useful for amplitude parameter, where a uniform prior may have previously been used. It has the property of being roughly
flat at low values of the parameter and with an exponetial fall off at large values. This gives the advantage over a uniform prior with a
sharp upper cut-off in that the distribution is continuous rather than (perhaps) arbitrarily truncated.

This form of prior was used for the gravitational wave amplitude parameter, $h_0$ in \citet{2017arXiv170107709T} as opposed to the earlier use of
a uniform prior with arbitrary upper cut-off used in, e.g., \citet{2014ApJ...785..119A}.

\subsection{Splitting the data}\label{sec:splitting}

The Student's $t$-likelihood function (\S\ref{sec:stlikelihood}) requires us to have an idea about the timescales on which the data is
stationary, i.e.\ periods when the noise in the data is best described as being drawn from a single
Gaussian distribution. We therefore use a scheme similar to the BayesianBlocks change point algorithm
\citep{1998ApJ...504..405S} to find points at which the statistics of the noise changes. This is done with
a top-down iterative {\it divide and conquer} approach \citep{2000physics...9033S}.

We start by taking a full complex time series data set for a detector and subtracting a running median from both
the real and imaginary components. This
subtraction is performed to try and remove the effect of any very strong signals in the data, as we want to
assess the properties of the noise rather than any signal.\footnote{In reality this is only likely to be a current issue for strong
simulated signals injected into the data, as real signals will have a very low signal-to-noise ratio in short stretches of data.} The running median is calculated over a running window
of 30 data points (or a minimum of 15 points at the start are end of the data), but not accounting for gaps in the data.
For the standard data sample rate of
1 per 60 seconds, this means that half an hour of data is used, over which time the detector antenna patterns will not change
too drastically (unless there is a gap in the data). However, it should be noted that if using much more slowly sampled data set
then this hardcoded 30 sample window may not work well for very strong signals.

The next thing we do is calculate the evidence that the whole (running-median-removed) dataset is drawn from a single
Gaussian distribution with unknown variance. To do this we just use Equation~\ref{eq:nulllike}, with $N_{\rm dets} =1$,
$N_{\rm s}=1$, $M_{1,1}=1$, $n_{1,1,0}=1$ and, given a dataset of length $N$, set $m_{1,1,1}=N$. We will call the natural logarithm
of this value $\ln{Z_{\rm single}}$. We want to calculate the odds between this evidence and that for the data containing
{\it any} single change point (i.e. point at which the data seems to be drawn from a different Gaussian distribution). So, for
one single change point, at index $i$, we calculate $\ln{Z_{\rm cp,i}}$, by using Equation~\ref{eq:nulllike} and setting
$N_{\rm dets} =1$, $N_{\rm s}=1$, $M_{1,1}=2$, $n_{1,1}=\{1,i\}$ and $m_{1,1} = \{i,N-i\}$. We also impose a minumum allowed data
chunk size of $n_{\min}$ (which defaults to a value of 5 data points in the code), so that $i \geqslant n_{\rm min}$. As we want
the evidence for {\it any} single change point, we must take the sum all the single change point evidences
\begin{equation}
 \ln{Z_{\rm any-cp}} = \ln{\left(\sum_{i=n_{\min}}^{N-n_{\rm min}} Z_{\rm cp,i} \right)}.
\end{equation}
Finally, we are left with the odds (assuming equal prior probabilities for each hypothesis)
\begin{equation}\label{eq:cpodds}
 \ln{\mathcal{O}_{\rm any-cp}^{\rm single}} = \ln{Z_{\rm any-cp}} - \ln{Z_{\rm single}},
\end{equation}
for which values above zero favour there being a change point in the data. We could simply use this as the criterion on which to
split the data into two chunks at the change point value $i$ with the largest evidence. However, in practice this leads to
data drawn from a single Gaussian distribution actually being split far too often, and also there being a dependence on the
data length. So, instead, we empirically calculate an odds threshold above which the value of Equation~\ref{eq:cpodds} must be to
impose a split in the data.\footnote{A different approach would be to apply different priors to each hypothesis. These are most
probably fairly straighforward to work out (i.e. something using the number of change point positions tried), although we have just taken
the empirical approach.} The empirical threshold we use is worked out by finding the 99\% upper limit on value of
Equation~\ref{eq:cpodds} when data is purely drawn from a single Gaussian distribution (i.e. it gives a 1\% false alarm probability)
as a function of the data length $N$. The results of doing this for 30\,000 instances of data for a range of $n$ values, and three
different false alarm probabilities (1\%, 0.5\% and 0.1\%), are shown in Figure~\ref{fig:changepoint}. Fitting a line to the 1\% false alarm
probability points gives and odds ratio threshold dependence of
\begin{equation}
 T = 4.07 + 1.33\log{}_{10}{N}.
\end{equation}

The above process just splits the data once, so it must be applied iteratively to split the data further. When a change point is
found then the process is repeated on the two split data chunks, and stops when either no change point is found within a chunk,
or the split would give a segment smaller than $n_{\rm min}$.

\subsection{Signal-to-noise ratio calculation}\label{sec:snr}

Whether a signal is found or not the code will output a signal-to-noise ratio (SNR). The calculated SNR uses the set of signal
parameters, $\vec{\theta}_{\rm ML}$, that give the maximum value for the likelihood, which are then used to create the best-fit
complex signal template $y(\vec{\theta}_{\rm ML})$. We then use calculations of the
noise standard deviation for each data chunk after removal of a running median from the data (the splitting and running median are
described in \S\ref{sec:splitting}). The SNR is then calculated as
\begin{equation}\label{eq:snr}
 \rho = \sqrt{\sum_{i=1}^{N_{\rm dets}} \sum_{j=1}^{N_{\rm s}}
\sum_{k=1}^{M_{i,j}} \sum_{n=n_{i,j,0}}^{n_{i,j,0}+(m_{i,j,k}-1)}\left(\frac{|y(\vec{\theta}_{\rm ML})_{i,j,n}|^2}{\sigma_{i,j,k}^2}\right)},
\end{equation}
using the notation given in \S\ref{sec:likelihood}.\footnote{If using data input from the spectral interpolation code \citep{2017CQGra..34a5010D}
the noise standard deviations are actually passed to the code (see \S\ref{app:examples}), so in Equation~\ref{eq:snr} $M_{i,j}$ is replaced
by $L_{i,j}$ and $y(\vec{\theta}_{\rm ML})_{i,j,n}$ is replace by $y(\vec{\theta}_{\rm ML})_{i,j,k}$ as in \S\ref{sec:glikelihood}.}

