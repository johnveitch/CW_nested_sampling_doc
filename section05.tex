\section{Evaluating the evidence calculation}

\subsection{Simulated Gaussian noise}

Here we will test accuracy and distribution of the odds ratios calculated by the nested sampling
algorithm.

\subsubsection{Odds ratio distribution}

The nested sampling algorithm is a stochastic sampling method for estimating the evidence integral
and as such the value returned will vary each time it is run. It will also be dependent on the
number of live points used in the calculation, with larger numbers of live points giving more
accurate and consistent answers. The uncertainty on a given evidence calculation can be calculated
analytically \citep[see][]{Skilling:2006} from the ``information'', but here we evaluate it
empirically.

We created a single complex data set consisting of 1440 points with the real and imaginary parts drawn
from a Gaussian distribution with $\mu = 0$ and $\sigma = 1\ee{-23}$. Our code was run on this data set
(for a source located at $\alpha = 0$ and $\delta = 0$) 125 times each using 128, 256, 512, 1024, 2048 
and 4096 live points. The code assumed stationary block sizes of 30 points (this is that assumed by
the legacy code, but does not matter much for this example). The code was also run using both the 
Student's {\it t} likelihood function in Eqn.~\ref{eq:stlikelihood} and the Gaussian likelihood 
function in Eqn.~\ref{eq:gausslikelihood} (for which the known $\sigma$ value of $1\ee{-23}$ was used).

\subsection{LIGO S5 data}

\subsubsection{Evaluating coherent signals}

We can run the code such that it provides the joint evidence for a signal in multiple detectors, but we can
also run it for individual detectors. This allows us to compare the model that there is a coherent signal
between detectors (as would be expected for a real signal) to the model that the signal is incoherent (i.e.\ has independent parameters)
between detectors (as you might expect from a detector line artefact). This can be done by comparing the joint
analysis evidence to the product of the individual detector evidences
\begin{equation}
\mathcal{O}_{\rm coherent} = \frac{\mathcal{Z}_{\rm Joint}}{\prod_{i=1}^{N_{\rm det}} \mathcal{Z}_{{\rm det}_i}}
\end{equation}

  
  